\section{Predictor-Corrector Method}
In this chapter, we introduce the predictor-corrector method\footnote{The name predictor-corrector originates from numerical methods for ordinary differential equations.}.
This method is widely adopted in many software packages as an practically efficient way to solve problems.
The predictor-corrector method was originally proposed by Mehrotra \cite{Mehrotra1992}.
The version presented by Wright \cite{Wright1997} has become widely used in the field.
Here, we introduce the formulation by Toh et al. \cite{toh1999}, which, although the process differs slightly, is equivalent to Wright's version.

The predictor-corrector method computes the search direction $(\Delta x, \Delta y, \Delta z)$ in two stages.
In the first stage (predictor step), we compute an affine scaling direction $(\delta x, \delta y, \delta z)$ by setting $\sigma = 0$, which aims to reduce the duality gap aggressively.
Based on the progress achieved by this predictor direction, we determine an appropriate centering parameter $\sigma$, and in the second stage (corrector step), we compute the final search direction $(\Delta x, \Delta y, \Delta z)$ that balances progress toward optimality with proximity to the central path.


\paragraph{Predictor Step}
First, set $\sigma=0$ to compute the affine scaling direction.
This replaces $R^p_{\text{comp}}$ in equation \eqref{eq:NewtonKKT} with
\[R^{p,\text{pred}}_{\text{comp}}=\nu^p \, e^p - (G^p x^p) \circ ((G^p)^{-1} z^p).\]
Solving equation \eqref{eq:NewtonKKT} with this modification yields the predictor direction $(\delta x, \delta y, \delta z)$.
Next, compute trial step sizes $\alpha_P, \alpha_D$ for this predictor direction using the method in Section~\ref{sec:step_size}, without actually updating the variables.
These step sizes are used to estimate the potential reduction in the duality gap and to determine an appropriate centering parameter $\sigma$.
Specifically, the centering parameter is computed as
\[
   \sigma=\min\left\{1, \left(\frac{\inprod{x + \alpha_P \delta x}{z + \alpha_D \delta z}}{\inprod{x}{z}}\right)^\psi\right\},
\]
where $\psi \ge 1$ is an exponent parameter.
Empirical studies suggest that $\psi \in \{2,3,4\}$ works well.
SDPT3 adaptively selects $\psi$ based on a default parameter $\hat{\psi}=3$:
\[
\psi = \begin{cases}
    \max\{\hat{\psi}, 3 \min(\alpha_P, \alpha_D)^2\} & \text{if} ~ \mu > 10^{-6}, \\
    \max\{1, \min\{\hat{\psi}, 3 \min(\alpha_P, \alpha_D)^2\}\} & \text{otherwise}.
\end{cases}
\]
Recall that $\mu$ is defined as in equation \eqref{eq:mu}.

\paragraph{Corrector Step}
With the computed centering parameter $\sigma$, we now solve for the final search direction.
The complementarity residual in equation \eqref{eq:NewtonKKT} is replaced with
\[R^{p,\text{corr}}_{\text{comp}}=\max\{\sigma \mu, \nu^p\} e^p - \big(G^p x^p\big) \circ\big ((G^p)^{-1} z^p\big) - \big(G^p \delta x^p\big)\circ\big((G^p)^{-1} \delta z^p\big),\]
where the last term is the second-order correction from the predictor step.
Solving equation \eqref{eq:NewtonKKT} with this modification yields the final search direction $(\Delta x, \Delta y, \Delta z)$.
We then compute the step sizes $\beta_P, \beta_D$ for this search direction using the method in Section~\ref{sec:step_size}, and update the solution as
\[(x^+, y^+, z^+) = (x, y, z) + (\beta_P \Delta x, \beta_D \Delta y, \beta_D \Delta z).\]

\medskip
\medskip
A key advantage of the predictor-corrector method is its computational efficiency.
Between the predictor and corrector steps, only the complementarity residual $R^p_{\text{comp}}$ changes, which affects only the right-hand side vector $h$ in the Schur complement system \eqref{eq:Schur_complement_Mat}.
The coefficient matrix $\mathcal{M}$ and its factorization (LU or Cholesky) computed in the predictor step can be reused in the corrector step.

Furthermore, the right-hand side vector can be efficiently updated.
Letting $h_{\text{pred}}$ and $h_{\text{corr}}$ denote the vectors for the predictor and corrector steps respectively, we have
\[h_{\text{corr}}=h_{\text{pred}} + \sum_{p\in P\setminus P^{\text{u}}} \mathcal{A}^p (\mathcal{E}^p)^{-1} \big((G^p \delta x^p) \circ ((G^p)^{-1} \delta z^p) \big).\]
Thus, although the predictor-corrector method requires solving the linear system twice, the additional computational cost is minimal due to these reuse strategies.
