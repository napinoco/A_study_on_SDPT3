\section{Predictor-Corrector Method}
In this chapter, we introduce the predictor-corrector method\footnote{The name predictor-corrector originates from numerical methods for ordinary differential equations.}.
This method is widely adopted in many software packages as an efficient way to solve problems.
The original method proposed by Mehrotra \cite{Mehrotra1992} is somewhat complex, so we introduce the version by Toh et al. \cite{toh1999} here.
Although the process differs slightly from the widely used algorithms (e.g., \cite{Wright1997}), the results obtained are equivalent.

In the predictor-corrector method, the search direction $(\Delta x, \Delta y, \Delta z)$ is computed in two stages.
Intuitively, in the first stage (predictor step), we compute a search direction $(\delta x, \delta y, \delta z)$ that reduces the duality gap.
Using this information, we determine a target point on the central path, and in the second stage (corrector step), we compute a search direction $(\Delta x, \Delta y, \Delta z)$ that pulls the solution back towards the central path.

\paragraph{Predictor Step}
Temporarily set $\sigma=0$, i.e., consider the case where
\[R^p_{\mathrm{comp}}=\nu^p \, e^p - (G^p x^p) \circ ((G^p)^{-1} z^p).\]
Let $(\delta x, \delta y, \delta z)$ be the solution to the equation \eqref{eq:NewtonKKT} in this case.
Using this predictor direction $(\delta x, \delta y, \delta z)$, compute the step sizes $\alpha_P, \alpha_D$ using the method in Section~\ref{sec:step_size}.
(This does not actually update the variables but temporarily determines $\alpha_P, \alpha_D$.)
Evaluate the estimated duality gap $\hat{\mu}$ using this predictor direction and step sizes, and determine the value of $\sigma$ to be used.
Specifically, with a parameter $\psi \ge 1$,
\[
   \sigma=\min\left\{1, \frac{\inprod{x + \alpha_P \delta x}{z + \alpha_D \delta z}}{\inprod{x}{z}}\right\}^\psi.
\]
Empirically, it is said that adopting $\psi=2,3,4$ is good, and SDPT3 uses a given parameter $\hat{\psi}=3$ to set
\[
\psi = \begin{cases}
    \max\{\hat{\psi}, 3 \min(\alpha_P, \alpha_D)^2\} & \text{if} ~ \mu > 10^{-6}, \\
    \max\{1, \min\{\hat{\psi}, 3 \min(\alpha_P, \alpha_D)^2\}\} & \text{otherwise}.
\end{cases}
\]
Here, $\mu$ is defined as in equation \eqref{eq:mu}.

\paragraph{Corrector Step}
Next, use the redefined $\sigma$. Furthermore, replace $R^p_{comp}$ in equation \eqref{eq:NewtonKKT} with
\[R^{p, corr}_{comp}=R^p_{comp}-(G^p \delta x^p)\circ((G^p)^{-1} \delta z^p),\]
and solve to obtain the final search direction $(\Delta x, \Delta y, \Delta z)$.
Compute the step sizes $\beta_P, \beta_D$ for this search direction using the method in Section~\ref{sec:step_size}.

\paragraph{Solution Update}
Update the next iteration point as $(x^+, y^+, z^+) = (x, y, z) + (\beta_P \Delta x, \beta_D \Delta y, \beta_D \Delta z)$.

In the equations to be solved between the predictor step and the corrector step, only $R^p_{\mathrm{comp}}$ changes.
That is, in the equation \eqref{eq:Schur_complement_Mat} to be solved, only the right-hand side vector $h$ changes.
Therefore, the coefficient matrix $\mathcal{M}$ computed in the predictor step, as well as its LU decomposition or Cholesky decomposition results, can be reused in the corrector step.
Moreover, for the value of $h$, let $h_{pred}$ be the one used in the predictor step and $h_{corr}$ be the one used in the corrector step.
Then,
\[h_{corr}=h_{pred} + \sum_{p\in P\setminus P^u} \mathcal{A}^p (\mathcal{E}^p)^{-1} \big((G^p \delta x^p) \circ ((G^p)^{-1} \delta z^p) \big),\]
so the computation result of the predictor step ($h_{pred}$) can be reused.
Thus, although introducing the predictor-corrector method increases the number of times the equation needs to be solved twice in succession, the increase in computational cost is relatively limited.


\medskip