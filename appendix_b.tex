\section{SDPT3 Input Data Format}
This section provides a detailed explanation of the input data format for SDPT3.
Since SDPT3 is implemented in MATLAB, we use MATLAB notation throughout this section:
\begin{itemize}
\item \texttt{[a,b]}: Horizontal concatenation of vectors or matrices \texttt{a} and \texttt{b}, i.e., $\begin{pmatrix} a \; b \end{pmatrix}$.
\item \texttt{[a;b]}: Vertical concatenation of \texttt{a} and \texttt{b}, i.e., $\begin{pmatrix} a \\ b \end{pmatrix}$.
\end{itemize}

SDPT3 is called as follows:
\begin{lstlisting}[language=Matlab]
[x, y, info] = sdpt3(blk, At, C, b, OPTIONS);
\end{lstlisting}

For problems (P) and (D), the SDPT3 input arguments are set as follows.

\begin{itemize}
    \item \texttt{blk} is a $p_{\max} \times 2$ cell array\footnote{MATLAB's cell array is a container that can hold different data types, similar to Python's lists.} that specifies the type and dimension of each subcone $\mathbb{K}^p$: 
\begin{lstlisting}[language=Matlab,escapechar=\@]
blk{p, 1} = @$\texttt{type}^p$@;  blk{p, 2} = @$n_p$@;
\end{lstlisting}
where the type identifier for each cone is defined as:
\[
\texttt{type}^p = \begin{cases}
\texttt{'s'} & \text{if } \mathbb{K}^p = \mathbb{S}_+^{n_p}, \\
\texttt{'q'} & \text{if } \mathbb{K}^p = \mathbb{Q}^{n_p}, \\
\texttt{'l'} & \text{if } \mathbb{K}^p = \mathbb{R}^{n_p}_+, \\
\texttt{'u'} & \text{if } \mathbb{K}^p = \mathbb{R}^{n_p}. 
\end{cases}
\]
% 
\item \texttt{At} is a $p_{\max} \times 1$ cell array which specifies the constraint matrices for each subcone: For semidefinite cones, 
\begin{lstlisting}[language=Matlab,escapechar=\@]
At{p} = [@$\operatorname{svec}(a^p_1), \operatorname{svec}(a^p_2), ..., \operatorname{svec}(a^p_m)$@];
\end{lstlisting}
For other cones,
\begin{lstlisting}[language=Matlab,escapechar=\@]
At{p} = [@$a^p_1, a^p_2, ..., a^p_m$@];
\end{lstlisting}
% 
\item \texttt{C} is a $p_{\max} \times 1$ cell array that stores the objective function coefficients for each subcone:
\begin{lstlisting}[language=Matlab,escapechar=\@]
  C{p} = @$c^p$@;
\end{lstlisting}
%
\item \texttt{b} is an $m \times 1$ vector representing the right-hand side of equality constraints:
\begin{lstlisting}[language=Matlab,escapechar=\@]
  b = [@$b_1; b_2; ...; b_m$@];
\end{lstlisting}
%
\item \texttt{OPTIONS} is a structure that allows various option settings. 
The coefficient $\nu^p$ can be set through \texttt{OPTIONS.parbarrier} which is an optional $p_{\max} \times 1$ vector:
\begin{lstlisting}[language=Matlab,escapechar=\@]
OPTIONS.parbarrier = [@$\nu^1; \nu^2; ...; \nu^{p_{\max}}$@];
\end{lstlisting}
\end{itemize}


\begin{example}
The following optimization problem demonstrates how to provide input to SDPT3:
\[
\left|
\begin{array}{cl}
\max  & 6y_1 + 4y_2 + 5y_3 \\
\mathrm{s.t.} 
& 16y_1 - 14y_2 + 5y_3 \leq -3 \\
& 7y_1 + 2y_2 \leq 5 \\
& \left\| \begin{pmatrix} 8y_1 + 13y_2 - 12y_3 - 2 \\ -8y_1 + 18y_2 + 6y_3 - 14 \\ y_1 - 3y_2 - 17y_3 - 13 \end{pmatrix} \right\|  \leq -24y_1 - 7y_2 + 15y_3 + 12 \\
& \left\| \begin{pmatrix} y_1 \\ y_2 \\ y_3 \end{pmatrix} \right\|  \leq 10 \\
& \begin{pmatrix} 
7y_1 + 3y_2 + 9y_3 & -5y_1 + 13y_2 + 6y_3 & y_1 - 6y_2 - 6y_3 \\
-5y_1 + 13y_2 + 6y_3 & y_1 + 12y_2 - 7y_3 & -7y_1 - 10y_2 - 7y_3 \\
y_1 - 6y_2 - 6y_3 & -7y_1 - 10y_2 - 7y_3 & -4y_1 - 28y_2 - 11y_3
\end{pmatrix}  \preceq \begin{pmatrix} 68 & -30 & -19 \\ -30 & 99 & 23 \\ -19 & 23 & 10 \end{pmatrix}
\end{array}
\right.
\]
This problem can be categorized as a dual problem (D) with:
\begin{itemize}
    \item $m = 3$, 
    $\quad b = \begin{pmatrix} 6,  4,  5 \end{pmatrix}^T$
    \item $p_{\max} = 4$, 
    $\quad \nu^p = 0$ for all $p=1,\ldots,4$
    \item $\mathbb{K}^1 = \mathbb{R}^{n_1}_+$, 
    $\mathbb{K}^2 = \mathbb{Q}^{n_2}$, 
    $\mathbb{K}^3 = \mathbb{Q}^{n_3}$, 
    $\mathbb{K}^4 = \mathbb{S}^{n_4}_+$
    \item $n_1 = 2$, 
    $n_2 = 4$, 
    $n_3 = 4$, 
    $n_4 = 3$
    \item $c^1 = \begin{pmatrix} -3 \\ 5 \end{pmatrix}$, 
    $a^1_1 = \begin{pmatrix} 16 \\ 7 \end{pmatrix}$, 
    $a^1_2 = \begin{pmatrix} -14 \\ 2 \end{pmatrix}$, 
    $a^1_3 = \begin{pmatrix} 5 \\ 0 \end{pmatrix}$
    \item $c^2 = \begin{pmatrix} 12 \\ -2 \\ -14 \\ -13 \end{pmatrix}$, 
    $a^2_1 = \begin{pmatrix} 24 \\ -8 \\ 8 \\ -1 \end{pmatrix}$, % y1
    $a^2_2 = \begin{pmatrix} 7 \\ -13 \\ -18 \\ 3 \end{pmatrix}$, % y2
    $a^2_3 = \begin{pmatrix} -15 \\ 12 \\ -6 \\ 17 \end{pmatrix}$ % y3
    \item $c^3 = \begin{pmatrix} 10 \\ 0 \\ 0 \\ 0 \end{pmatrix}$, 
    $a^3_1 = \begin{pmatrix} 0 \\ -1 \\ 0 \\ 0 \end{pmatrix}$, 
    $a^3_2 = \begin{pmatrix} 0 \\ 0 \\ -1 \\ 0 \end{pmatrix}$, 
    $a^3_3 = \begin{pmatrix} 0 \\ 0 \\ 0 \\ -1 \end{pmatrix}$
    \item $c^4 = \begin{pmatrix} 68 & -30 & -19 \\ -30 & 99 & 23 \\ -19 & 23 & 10 \end{pmatrix}$, \\
    $a^4_1 = \begin{pmatrix} 7 & -5 & 1 \\ -5 & 1 & -7 \\ 1 & -7 & -4 \end{pmatrix}$, 
    $a^4_2 = \begin{pmatrix} 3 & 13 & -6 \\ 13 & 12 & -10 \\ -6 & -10 & -28 \end{pmatrix}$, 
    $a^4_3 = \begin{pmatrix} 9 & 6 & -6 \\ 6 & -7 & -7 \\ -6 & -7 & -11 \end{pmatrix}$
\end{itemize}

The SDPT3 input becomes:
\begin{lstlisting}[language=Matlab]
blk{1,1} = 'l'; blk{1,2} = 2;
blk{2,1} = 'q'; blk{2,2} = 4;
blk{3,1} = 'q'; blk{3,2} = 4;
blk{4,1} = 's'; blk{4,2} = 3;

At{1} = [16, -14, 5;
         7,   2, 0];

At{2} = [24,   7, -15;
         -8, -13,  12;
          8, -18,  -6;
         -1,   3,  17];

At{3} = [ 0,  0,  0;
         -1,  0,  0;
          0, -1,  0;
          0,  0, -1];

A1_sdp = [7, -5, 1; -5, 1, -7; 1, -7, -4];
A2_sdp = [3, 13, -6; 13, 12, -10; -6, -10, -28];
A3_sdp = [9, 6, -6; 6, -7, -7; -6, -7, -11];
pblk = {'s', 3};
At{4} = [svec(pblk,A1_sdp), svec(pblk,A2_sdp), svec(pblk,A3_sdp)];
%% that is equivalent to:
% At(4) = svec(pblk, {A1_sdp, A2_sdp, A3_sdp});
%% or explicitly:
% s = sqrt(2);
% At{4} = [ 7  ,   3  ,   9  ;
%          -5*s,  13*s,   6*s;
%           1  ,  12  ,  -7  ;
%           1*s,  -6*s,  -6*s;
%          -7*s, -10*s,  -7*s;
%          -4  , -28  , -11  ];

C{1} = [-3; 5];
C{2} = [12; -2; -14; -13];
C{3} = [10; 0; 0; 0];
C{4} = [68, -30, -19; -30, 99, 23; -19, 23, 10];

b = [6; 4; 5];

[x, y, info] = sdpt3(blk, At, C, b);
\end{lstlisting}
\end{example}