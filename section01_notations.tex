\section{Notations}\label{sec:notation}


For a positive integer $n$, let $\mathbb{R}^n$ denote the $n$-dimensional real vector space.
In this paper, we treat elements $x$ of $\mathbb{R}^n$ as column vectors, with the $i$-th component denoted by $x_i$.

For positive integers $m$ and $n$, let $\mathbb{R}^{m \times n}$ denote the space of $m \times n$ real matrices.
The $(i,j)$ component of $x \in \mathbb{R}^{m \times n}$ is denoted by $x_{ij}$.
For both vectors and matrices, the transpose of $x$ is denoted by $x^T$.
If $x \in \mathbb{R}^{n \times n}$ is invertible, its inverse is denoted by $x^{-1}$, and the transpose of the inverse by $x^{-T}$.
The trace of a matrix $x$ is denoted by $\operatorname{trace}(x)$, and the determinant by $\operatorname{det}(x)$.
For a vector $x \in \mathbb{R}^n$, we denote by $\operatorname{diag}(x) \in \mathbb{R}^{n \times n}$ the diagonal matrix with diagonal entries given by the components of $x$:
\[
\operatorname{diag}(x) = \begin{pmatrix}
x_1 & 0 & \cdots & 0 \\
0 & x_2 & \cdots & 0 \\
\vdots & \vdots & \ddots & \vdots \\
0 & 0 & \cdots & x_n
\end{pmatrix}.
\]
The identity matrix of size $n \times n$ is denoted by $I_n$ or simply $I$ when the dimension is clear from context.
Additionally, we define the diagonal matrix $J$ as follows:
\[
    J = \begin{pmatrix}
        1 & 0_{n-1}^T \\
        0_{n-1} & -I_{n-1}
    \end{pmatrix} \in \mathbb{R}^{n \times n},
\]
where $0_{n-1} \in \mathbb{R}^{n-1}$.

Regarding the notation $\|a\|$, if $a$ is a matrix, it represents the Frobenius norm, and if $a$ is a vector, it represents the L2 norm (Euclidean norm).
Specifically,
\[
\|a\| = 
\begin{cases}
    \sqrt{\sum_{i=1}^m \sum_{j=1}^n a_{ij}^2} & \text{if } a \in \mathbb{R}^{m \times n},\\
    \sqrt{\sum_{i=1}^n a_i^2} & \text{if } a \in \mathbb{R}^n.
\end{cases}
\]

We define several sets as follows:
\begin{itemize}
    \item 
    \textbf{$n$-dimensional non-negative real cone:} \\
    \[
      \mathbb{R}^n_+ 
      = \{\,x \in \mathbb{R}^n \mid x \geq 0 \}.
    \]

    \item 
    \textbf{$n$-dimensional second-order cone:} \\
    \[
      \mathbb{Q}^n 
      = \left\{ \begin{pmatrix} x_0 \\ \bar{x} \end{pmatrix} \in \mathbb{R}^n 
         \;\middle|\; x_0 \in \mathbb{R}^1_+, \; \bar{x} \in \mathbb{R}^{n-1}, \; \|\bar{x}\| \le x_0 \right\}.
    \]
    Here, we define a non-negative real-valued function $\gamma : \mathbb{Q}^n \to \mathbb{R}^1_+$ as
    \[
      \gamma(x) = \sqrt{x^T J\, x}.
    \]

    \item 
    \textbf{Space of $n$-dimensional real symmetric matrices:} \\
    \[
      \mathbb{S}^n = \{\, x \in \mathbb{R}^{n \times n} \mid x = x^T \}.
    \]

    \item 
    \textbf{$n$-dimensional real positive semi-definite cone:} \\
    \[
      \mathbb{S}^n_+ 
      = \{\, x \in \mathbb{S}^n \mid a^T x\, a \ge 0 \;\; \forall a \in \mathbb{R}^n \}.
    \]
\end{itemize}

For $x \in \mathbb{R}^n$ and $\epsilon > 0$, the open ball of radius $\epsilon$ centered at $x$ is defined as
\[
  B(x, \epsilon) = \{ y \in \mathbb{R}^n \mid \|y - x\| < \epsilon \}.
\]
For a set $S \subseteq \mathbb{R}^n$, the affine hull of $S$ is defined as
\[
  \operatorname{aff}(S) = \left\{ \sum_{i=1}^k \lambda_i x_i \mid x_i \in S, \lambda_i \in \mathbb{R}, \sum_{i=1}^k \lambda_i = 1, k \in \mathbb{N} \right\}.
\]
The interior of a set $S$ is defined as
\[
  \operatorname{int}(S) 
  = \{\, x \in S \mid \exists\, \epsilon > 0 \quad \text{s.t.} \quad B(x, \epsilon) \subseteq S \},
\]
whereas the relative interior is defined as the interior within the affine hull of $S$:
\[
\operatorname{relint}(S)
=\{\,
x \in S 
\mid \exists\, \epsilon > 0 \quad \text{s.t.} \quad \bigl(\operatorname{aff}(S)\cap B(x,\epsilon)\bigr)\,\subseteq\,S
\}.
\]
