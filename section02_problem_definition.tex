\section{Problem Definition}

In this paper, we consider the following primal problem \((P)\) and its corresponding dual problem \((D)\):
\begin{equation*}
    \everymath{\displaystyle}
    \renewcommand{\arraystretch}{2.0}
    (P)~
    \left|
    \begin{array}{cl}
         \min_{x} & \displaystyle 
             \sum_{p \in P} \inprod{c^p}{x^p}_p 
             \;+\;\sum_{p \in P} \phi^p\bigl(x^p;\,\nu^p\bigr) \\[3pt]
         \text{s.t.} 
         & \displaystyle 
             \sum_{p \in P} \inprod{a^p_{k}}{x^p}_p 
             \;=\; b_k \quad (k= 1,2,\ldots,m), \\[3pt]
         & x^p \;\in\; \mathbb{K}^p \quad (p\in P),
    \end{array}
    \right.
    \qquad
    (D)~
    \left|
    \begin{array}{cl}
         \max_{y,z} & \displaystyle 
            \sum_{k=1}^m b_k\, y_k 
            \;+\;\sum_{p \in P} \phi^{p*}\bigl(z^p;\,\nu^p\bigr) \\[3pt]
         \text{s.t.} 
         & \displaystyle 
            c^p \;-\; \sum_{k=1}^m a^p_k\,y_k \;=\; z^p \quad (p\in P), \\[3pt]
         & y \;\in\; \mathbb{R}^m, \quad z^p \;\in\; (\mathbb{K}^p)^* \quad (p\in P)
    \end{array}
    \right.
\end{equation*}
where the following parameters are assumed to be given:
\begin{itemize}
    \item A positive integer $p_{\max}$ representing the number of cone blocks, and the index set $P = \{1, 2, \ldots, p_{\max}\}$.
    \item For each block $p \in P$, a positive integer $n_p$ representing the dimension of the variables.
    \item A positive integer $m$ representing the number of constraints.
    \item The cone $\mathbb{K}^p$ for each $p \in P$, which is one of $\mathbb{S}^{n_p}_+$, $\mathbb{Q}^{n_p}$, $\mathbb{R}^{n_p}_+$, or $\mathbb{R}^{n_p}$.
    \item Coefficients $c^p$ and $a^p_k$ for $k = 1, \ldots, m$ and $p \in P$, where:
      \begin{itemize}
          \item $c^p, a^p_k \in \mathbb{S}^{n_p}$ if $\mathbb{K}^p = \mathbb{S}^{n_p}_+$
          \item $c^p, a^p_k \in \mathbb{R}^{n_p}$ otherwise
      \end{itemize}
    \item A coefficient vector $b \in \mathbb{R}^m$.
    \item Non-negative parameters $\nu^p \ge 0$ for all $p \in P$.
\end{itemize}
%
The notation $(\mathbb{K}^p)^*$ denotes the dual cone of $\mathbb{K}^p$, 
i.e., $(\mathbb{K}^p)^* = \{ z^p \mid \inprod{x^p}{z^p}_p \geq 0 \text{ for all } x^p \in \mathbb{K}^p \}$, 
which in our case takes the following forms:
\[
(\mathbb{K}^p)^* = 
\begin{cases}
    \mathbb{S}_{+}^{n_p} & \text{if } \mathbb{K}^{p} = \mathbb{S}_{+}^{n_p},\\
    \mathbb{Q}^{n_p}     & \text{if } \mathbb{K}^{p} = \mathbb{Q}^{n_p},\\
    \mathbb{R}_{+}^{n_p} & \text{if } \mathbb{K}^{p} = \mathbb{R}_{+}^{n_p},\\
    \{0\}                & \text{if } \mathbb{K}^{p} = \mathbb{R}^{n_p}.
\end{cases}
\]
%
The inner product $\left\langle a, x \right\rangle_p$ is defined as:
\[
\left\langle a, x \right\rangle_p = 
\begin{cases}
    \operatorname{trace}(a^T x) 
    = \sum_{i=1}^{n_p} \sum_{j=1}^{n_p} a_{ij} x_{ij}, 
    & \text{if } \mathbb{K}^p = \mathbb{S}_{+}^{n_p},\\[6pt]
    a^T x = \sum_{i=1}^{n_p} a_i x_i, 
    & \text{otherwise}.
\end{cases}
\]
%
The function $\phi^p : \mathbb{K}^p \to \mathbb{R}_+$ is a barrier function defined as:
\[
\phi^p(x^p;\, \nu^p) =
\begin{cases}
    -\nu^p \log \det(x^p), & \text{if } \mathbb{K}^{p} = \mathbb{S}_{+}^{n_p},\\
    -\nu^p \log \gamma(x^p), & \text{if } \mathbb{K}^{p} = \mathbb{Q}^{n_p},\\
    -\sum_{i=1}^{n_p} \nu^p \log x^p_i, & \text{if } \mathbb{K}^{p} = \mathbb{R}_{+}^{n_p},\\
    0, & \text{if } \mathbb{K}^{p} = \mathbb{R}^{n_p},
\end{cases}
\]
where $\gamma(x^p) = \sqrt{(x^p)^T J x^p}$ as introduced in Section~\ref{sec:notation}.
%
The notation $\phi^{p*}$ denotes the convex conjugate of $\phi^p$, 
i.e., $\phi^{p*}(z^p; \nu^p) = \sup_{x^p \in \mathbb{K}^p} \{\langle z^p, x^p \rangle_p - \phi^p(x^p; \nu^p)\}$, 
which in our case takes the following forms:
\[
\phi^{p*}(z^p;\, \nu^p) =
\begin{cases}
    \nu^p \log \det(z^p) + n_p \nu^p (1 - \log \nu^p), 
    & \text{if } \mathbb{K}^{p} = \mathbb{S}_{+}^{n_p},\\[4pt]
    \nu^p \log \gamma(z^p) + \nu^p (1 - \log \nu^p), 
    & \text{if } \mathbb{K}^{p} = \mathbb{Q}^{n_p},\\[4pt]
    \sum_{i=1}^{n_p} \left( \nu^p \log z^p_i + \nu^p (1 - \log \nu^p) \right), 
    & \text{if } \mathbb{K}^{p} = \mathbb{R}_{+}^{n_p},\\[3pt]
    0, & \text{if } \mathbb{K}^{p} = \mathbb{R}^{n_p}.
\end{cases}
\]

When $\nu^p = 0$ for all $p \in P$, problems $(P)$ and $(D)$ reduce to the standard form of conic linear programming.
By considering the case $\nu^p > 0$, we can handle a broader class of problems, including log-determinant optimization problems.